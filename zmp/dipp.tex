\pdfoutput=1
\documentclass[twoside,12pt]{article}%
\usepackage[utf8]{inputenc} % použito v případě jiného kódování
% aktuální kódování: utf8
\usepackage{dipp}
\usepackage[czech]{babel}
\usepackage{graphicx}
\begin{document}

\def\,{\penalty10000\hskip.25em}
\pagestyle{headings}

\cislovat{2}
\bakalarska

\titul{Vývoj aplikace na kontrolování změn na webových stránkách}{Jiří Kalvoda}{Mgr. Marek Blaha}{Blansko 2020}

\podekovani{TODO poděkování}

\prohlaseni{Prohlašuji, že jsem tuto práci vyřešil samostatně
s~použitím literatury, kterou uvádím v~seznamu}{V~Blansku dne \today}

\abstract{Kalvoda, J. }{Development an aplication for checking changes on web pages}{abstract}

\abstrakt{Kalvoda, J. \NAZEV}
{Tato závěrečná práce popisuje vývoj a použití aplikace na monitorování změn na webových stránkách.
Aplikace je vyvíjena v jazyce c++ pomocí knihovny Qt. Díky tomu se jedná o multiplatformní software.
Je dostupná včetně zdrojového kódu pod licencí GNU LGPL. Tato závěrečná práce obsahuje popis jejího fungování, implementace a použitého softwaru při jejím vývoji.}

\obsah

\kapitola{Úvod a~cíl práce}
\sekce{Úvod do problematiky}

Webové stránky se mohou neustále měnit, proto je dobré automaticky monitorovat jejich aktualizace.
Tato aplikace umožňuje automatizovat tento problém a tím uživateli ušetřit čas a eliminovat lidský chybový faktor.

Aplikace podporuje různé tolerance při načítání a porovnávání změn stránek.
Například na stránce, kde se část neustále mění, mohu tuto část vypustit, nebo přímo porovnávat jen nějaké části a podobně. Díky práci s cookies je možné navázat i složitější spojení se serverem a provést definovanou sekvenci úkolů (např. přihlásit se a načíst nějaký soukromý obsah). Historie stránek se může ukládat a pak lze v~ní vyhledávat a zjišťovat rozdíly mezi verzemi pomocí grafického porovnávání napojeného na uživatelův oblíbený prohlížeč.

\sekce{Cíl práce}

Cílem této práce je vyvinout funkční aplikaci umožnující zjišťování aktualizací, archivaci a porovnávání webových stránek (případně i jiných dokumentů) a publikovat ji cílovým uživatelům na různých operačních systémech.
K aplikaci také bude vypracována rozsáhlá uživatelská i technická dokumentace, která umožní její další vývoj.
Umožním tedy dalším programátorům tuto aplikaci pohodlně modifikovat a upravovat dle svých potřeb.
Cílem této práce je také aplikaci rozšířit mezi skupinu testovacích uživatelů a použít jejich připomínky a problémy k dalšímu vývoji a stabilizaci aplikace.

\kapitola{Přehled literatury}

\kapitola{Popis fungování a ovládání aplikace}
\sekce{Instalace}
\podsekce{Linux}
\podsekce{Windows}
\podsekce{macOS}

\sekce{Seznam stránek na kontrolu}
\podsekce{Adresa stránek}
\podsekce{způsoby kontroly změn}
\podsekce{Skupiny stránek, přihlášení a práce s cookie}

\sekce{Spuštění kontroly, tabulka změn, informační konzole}

\sekce{Otevření lokální kopie stránky, její napojení}

\sekce{Historie změn a její procházení}

\sekce{Grafické porovnávání verzí stránek}

\sekce{Klávesové zkratky}

\kapitola{Implementace aplikace}

\sekce{Použitý software}
\podsekce{Qt}
\podsekce{Git}

\sekce{Objektový model, rozdělení problému}
\sekce{Pozadí aplikace}

\sekce{Grafické uživatelské rozhraní}


\end{document}
