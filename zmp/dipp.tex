\pdfoutput=1
\documentclass[twoside,12pt]{article}%
\usepackage[utf8]{inputenc} % použito v případě jiného kódování
% aktuální kódování: utf8
\usepackage{dipp}
\usepackage[czech]{babel}
\usepackage{graphicx}


% ~~~~~ PACKAGE ~~~~~
\usepackage[utf8]{inputenc}  % základní kódování
\usepackage[czech]{babel}    % české nastavení
\usepackage{amsmath, amssymb}   % matematické simboly
% amsfonts
\usepackage{graphicx}  % barvy a jiné
\usepackage{lastpage}   % zjištění počtu stránek
\usepackage{calc}  % snazší výpočty délek
\usepackage{refcount} % převod reference na číslo - dvojstraný režim - dělení 2
\usepackage{fancyhdr} % hlavičky
\usepackage{etoolbox}  % změna geimetrie na určité stránky
\usepackage{pgfplots} % geogebra
\usepackage[final]{pdfpages} % geogebra v pdf
%mathrsfs, pdf, tikz, mathrsfs
\usepackage{listingsutf8} % zdrojové kódy
\usepackage{hyperref}
% ~~~~~ MATEMATICKÉ ~~~~~
\renewcommand{\(}{\left(}
\renewcommand{\)}{\right)}
\renewcommand{\[}{\left[}
\renewcommand{\]}{\right]}
\def\MATHINFO{}
  \newcommand{\mathinfo}[2]{\if ^\MATHINFO^ \underbrace{\def\MATHINFO{1} {#1} \def\MATHINFO{} }_{\text{#2}} \else \overbrace{\def\MATHINFO{} {#1} \def\MATHINFO{1}}^{\text{#2}} \fi}
\newcommand{\Mathinfo}[1]{\qquad \qquad \text{({#1})}}

\newcommand{\tg}{\mathrm{tg}}
\newcommand{\cotg}{\mathrm{cotg}}
\newcommand{\sgn}{\mathrm{sgn}}
\renewcommand{\angle}{\sphericalangle}
\newcommand{\degre}{\ensuremath{^\circ}}
\def\d{\degre}
\newcommand{\pri}{\overleftrightarrow}
\newcommand{\ppri}{\overrightarrow}
\def\imp{\Rightarrow}
\def\Imp{\quad\imp\quad}
\def\ekv{\Leftrightarrow}
\def\Ekv{\quad\ekv\quad}
\def\rimp{\Leftarrow}
\def\RImp{\quad\rimp\quad}
\renewcommand{\*}{\cdot{}}
\def\N{\mathbb{N}}
\def\Z{\mathbb{Z}}
\def\Q{\mathbb{Q}}
\def\R{\mathbb{R}}
\def\C{\mathbb{C}}
\def\E{\mathbb{E}}
\def\P{\mathbb{P}}
\def\US{\leftrightarrow{}}

\newcommand{\eqnl}[1][]{{#1}$$\\[-16px]$${}{#1}}

\newcommand{\derivation}[2]{\frac{\partial}{\partial {#1}}\({#2}\)}
\renewcommand{\j}[2][]{\ \mathrm{\if ^#1^ {#2} \else \frac{{#1}}{{#2}} \fi}}

\def\f{\frac}
\def\uv#1{„#1“}

\newcounter{lemmaCounter}[section]
\renewcommand{\thelemmaCounter}{\arabic{lemmaCounter}}
\newtheorem{lemmaBlock}[lemmaCounter]{}
\newenvironment{lemma}[3][]
{

	\refstepcounter{lemmaCounter}
		\label{#2}
	\textbf{Lemma \thelemmaCounter\if^#1^\else \ ({#1})\fi:}
	\emph{{#3}}\\{}
	Důkaz: 
}
{ 
\hfill $\Box$

}

% ~~~~~ INFORMATICKÉ ~~~~~
\newcommand{\refcodeline}[1]{{\footnotesize Viz \ref{#1}. řádek kódu na straně \pageref{#1}.}}
\newcommand{\refcodeblock}[1]{{\footnotesize Viz \ref{#1}.-\ref{#1-end}. řádek kódu na straně \pageref{#1}.}}
\definecolor {lightGrey}{RGB}{250,250,250}
\newcommand{\Onotation}[1]{\ifmmode\mathcal{O}({#1})\else$\mathcal{O}({#1})$\fi}
\renewcommand{\O}{\Onotation}
\newcommand{\codeF}[1]{\lstinputlisting[ numbers = left, numberstyle = \tiny ,frame = shadowbox, backgroundcolor =\color{lightGrey}, showstringspaces=false]{#1} }
\newcommand{\codeMain}{\codeF{main.cpp}}
\lstset{basicstyle =  \small \ttfamily, keywordstyle = \emph,commentstyle = \rmfamily ,  backgroundcolor =\color{lightGrey}, showstringspaces=false, escapeinside ={///}{/}, extendedchars=true, 
literate={á}{{\'a}}1 {í}{{\'i}}1 {é}{{\'e}}1 {ý}{{\'y}}1 {ú}{{\'u}}1 {ó}{{\'o}}1 {ě}{{\v{e}}}1 {š}{{\v{s}}}1 {č}{{\v{c}}}1 {ř}{{\v{r}}}1 {ž}{{\v{z}}}1 {ď}{{\v{d}}}1 {ť}{{\v{t}}}1 {ň}{{\v{n}}}1 {ů}{{\r{u}}}1 {Á}{{\'A}}1 {Í}{{\'I}}1 {É}{{\'E}}1 {Ý}{{\'Y}}1 {Ú}{{\'U}}1 {Ó}{{\'O}}1 {Ě}{{\v{E}}}1 {Š}{{\v{S}}}1 {Č}{{\v{C}}}1 {Ř}{{\v{R}}}1 {Ž}{{\v{Z}}}1 {Ď}{{\v{D}}}1 {Ť}{{\v{T}}}1 {Ň}{{\v{N}}}1 {Ů}{{\r{U}}}1  ,}
\def\c{\lstinline}





\begin{document}

\def\obrazek#1#2#3{
	\begin{figure}[tbhp]
  \centering
	{#1}
	\caption{{#2}}
  \label{fig:#3}
\end{figure}
	}
\def\obr#1#2{\obrazek{\includegraphics[width=9cm]{screenshots/#1.png}}{#2}{#1}}
\def\obrsize#1#2#3{\obrazek{\includegraphics[width=#3cm]{screenshots/#1.png}}{#2}{#1}}
\def\,{\penalty10000\hskip.25em}
\pagestyle{headings}

\cislovat{2}
\bakalarska


\titul{Vývoj aplikace na kontrolování změn na webových stránkách}{Jiří Kalvoda}{Mgr. Marek Blaha}{Blansko 2020}

\podekovani{TODO poděkování}

\prohlaseni{Prohlašuji, že jsem tuto práci vyřešil samostatně
s~použitím literatury, kterou uvádím v~seznamu}{V~Blansku dne \today}

\abstract{Kalvoda, J. }{Development an aplication for checking changes on web pages}{abstract}

\abstrakt{Kalvoda, J. \NAZEV}
{Tato závěrečná práce popisuje vývoj a použití aplikace na monitorování změn na webových stránkách.
Aplikace je vyvíjena v jazyce c++ pomocí knihovny Qt. Díky tomu se jedná o multiplatformní software.
Je dostupná včetně zdrojového kódu pod licencí GNU LGPL. Tato závěrečná práce obsahuje popis jejího fungování, implementace a použitého softwaru při jejím vývoji.}

\obsah

\kapitola{Úvod a~cíl práce}
\sekce{Úvod do problematiky}

Webové stránky se mohou neustále měnit, proto je dobré automaticky monitorovat jejich aktualizace.
Tato aplikace umožňuje automatizovat tento problém a tím uživateli ušetřit čas a eliminovat lidský chybový faktor.

Aplikace podporuje různé tolerance při načítání a porovnávání změn stránek.
Například na stránce, kde se část neustále mění, mohu tuto část vypustit, nebo přímo porovnávat jen nějaké části a podobně. Díky práci s cookies je možné navázat i složitější spojení se serverem a provést definovanou sekvenci úkolů (např. přihlásit se a načíst nějaký soukromý obsah). Historie stránek se může ukládat a pak lze v~ní vyhledávat a zjišťovat rozdíly mezi verzemi pomocí grafického porovnávání napojeného na uživatelův oblíbený prohlížeč.

\sekce{Cíl práce}

Cílem této práce je vyvinout funkční aplikaci umožnující zjišťování aktualizací, archivaci a porovnávání webových stránek (případně i jiných dokumentů) a publikovat ji cílovým uživatelům na různých operačních systémech.
K aplikaci také bude vypracována rozsáhlá uživatelská i technická dokumentace, která umožní její další vývoj.
Umožním tedy dalším programátorům tuto aplikaci pohodlně modifikovat a upravovat dle svých potřeb.
Cílem této práce je také aplikaci rozšířit mezi skupinu testovacích uživatelů a použít jejich připomínky a problémy k dalšímu vývoji a stabilizaci aplikace.
\obr{basic_app_small}{Základní okno aplikace.}


\kapitola{Popis fungování a ovládání aplikace}

\sekce{Instalace}
\podsekce{Kompilace ze zdrojového kódu}
Způsobem, jak aplikaci nainstalovat na většině používaných operačních systémů, je kompilace ze zdrojového kódu.
Zdrojové kódy aktuální stabilní verze je možné stá\-hnout z~\url{gitlab.com/JiriKalvoda/webupdatingindicator/tree/master}.
V~pří\-padě, že má uživatel zájem o~aktuálně nejnovější funkce, je možné použít testovací verzi produktu dostupnou z~\url{gitlab.com/JiriKalvoda/webupdatingindicator/tree/Test}.
Soubory lze stáhnout pomocí webového rozhraní a nebo je lze naklonovat s~použitím gitu.
Aplikaci pak lze zkompilovat za použití Qt knihoven.
Nejsnazší způsob je využít aplikace Qt Creator. Pomocí ní stačí otevřít soubor \c|WebUpdatingIndicator.pro| a v~levém rohu aplikace kliknout na tlačítko pro kompilaci.
Tímto způsobem by měla vzniknout samostatně spustitelná aplikace, kterou stačí umístit do požadované složky a v~ní ji spouštět.
Podporovaná by měla být libovolná verze Qt větší než 5.4.
Pro vývoj se používá Qt 5.12.5.

V~případě, že uživatel nechce provádět kompilaci ze zdrojového kódu, pro základní architektury a operační systémy je možné využít již zkompilované varianty.
Ty jsou dostupné na adrese \url{gitlab.com/JiriKalvoda/webupdatingindicator-install/tree/master}
\podsekce{Linux}
Na operačních systémech postavených na jádře Linuxu stačí pouze stáhnout a rozzipovat složku s~programem do uživatelem zvoleného adresáře.
Pracovní adresář aplikace je pak ten, ze kterého se aplikace spouští (nemusí tedy být shodný s~adresářem, ve kterém je umístěna aplikace).
Pro jednodušší spouštění je dobré vytvořit bash script, který bude obsahovat přepnutí polohy do pracovního adresáře a spuštění aplikace.
Příklad takového skriptu je uložen ve složce, ve které je umístěn program, pod názvem \c|run.sh|.
Pro možnost spouštění aplikace z~menu či pomocí přímého příkazu je možné tento skript umístit do adresáře \c|usr/bin|.

V~případě užívání správce oken i3 je vhodné nastavit, aby se okna porovnávání stránek zobrazovala jako plovoucí.
Toho lze docílit přidáním řádku \c|for_window [title="WebUpdatingIndicator compare"] floating enable| do konfiguračního souboru i3 umístěného v~\c|~/.config/i3/config|.
Pro snazší spouštění aplikace je také vhodné nadefinovat klávesovou zkratku.
Případně je možně vyhradit aplikaci speciální pracovní plochu a definovat její spuštění a přepnutí na danou plochu pomocí příkazů (nastaví spuštění na \c|$mod+Shift,| a zobrazení na \c|$mod+,|):\\
\begin{tabular}{l}
\c|bindsym $mod+Shift+comma workspace WUI;exec WebUpdatingIndicator.sh|\\
\c|bindsym $mod+comma workspace WUI|\\
\end{tabular}
\obr{linux-i3-2}{Aplikace otevřená v~operačním systému Linux Mint 19 \protect\linebreak se správcem oken i3.}
\obr{linux-cinnamon}{Aplikace otevřená v~operačním systému Linux Mint 19 se správcem oken Cinnamon.}

\podsekce{Windows}
Pro Windows existuje alternativní instalační složka. Jejím stažením a rozzipováním do uživatelem zvolené složky vznikne spustitelná aplikace. 
Jelikož na tomto operačním systému nejsou běžně dostupné potřebné dynamicky linkované knihovny Qt,
instalační složka je přímo obsahuje.
Důsledkem této skutečnosti je, že aplikace zabírá mnohem více diskového prostoru.
Dále je dobré vytvořit zástupce, přes kterého se bude daný program spouštět.
Při jeho vytváření se dá zvolit i pracovní adresář.
Jako ikonku je možné nastavit \c|Logo.ico|.
Zástupce je pak možné umístit například na plochu nebo do Start menu.
\obr{windows}{Aplikace otevřená v~operačním systému Windows 10.}

\podsekce{macOS}
Na tomto operačním systému je momentálně možná instalace pouze pomocí kompilace ze zdrojového kódu.


% TODO při inicializaci udělat ty složky.


\sekce{Seznam stránek na kontrolu}
Při spuštění si aplikace načte seznam stránek ke kontrole.
Ten je obsažen v souboru \c|pages.json|, který musí být umístěn v adresáři aplikace (respektive v adresáři, kde se aplikace spouští).
Soubor musí být validní json.
V případě, že soubor neexistuje nebo není validní, aplikace vypíše upozornění. % TODO naimplementovat to
Očekává se, že soubor obsahuje pole struktur.
Každá z nich obsahuje informace o jedné stránce, která se má kontrolovat.
\podsekce{Adresa stránky a název}
Každou stránku je nutné pojmenovat jednoznačným identifikátorem.
Proto je nutné u každé stránky definovat položku označenou klíčem \c|name|.
Toto jméno se pak zobrazuje v seznamu změn a také je nutné při vyhledávání v historii.
Jméno se také používá jako identifikátor v databázi i jako identifikátor souborů jednotlivých verzí stránky.

Dále je nutné u každé stránky definovat její umístění na webu, tedy její adresu.
K tomu slouží položky \c|server|, \c|dir|, \c|file|.
Ovšem není zapotřebí definovat všechny tyto položky.
Pod klíčem \c|server| by měla být definována adresa serveru, na kterém je požadovaná stránka včetně přístupového protokolu.
Tedy například \c|https://is.jaroska.cz| nebo pomocí ip adresy může zápis vypadat následovně \c|http://195.178.65.1|
V případě, že není požadovaná stránka na serveru umísténa v jeho kořenové složce, je doporučeno název složky (popřípadě celou cestu několika zanořených složek) umístit do položky \c|dir|.
\c|file| obsahuje jméno požadovaného souboru včetně přípony
V případě, že je požadován přístup základní soubor (index) na dané adrese, není potřeba \c|file| uvádět.
Dále zde také můžou být obsaženy parametry stránky, které se předávají metodou \c|GET|.
Stačí je uvézt za otazník.
Takový zápis může vypadat následovně: \c|index.php?akce=42&akcicka=0|.

V případě, že je nutné předat stránce informace pomocí protokolu \c|POST| (typicky při přihlašování na stránky), 
je možné v zápisu stránky užít klíče \c|post|.
%\c|formUsername=HugoKokoska&formPassword=BezpecneHeslo|.

Nejjednodušší způsob, jak zjistit tyto informace pro požadovanou stránku je využít prohlížeč.
Většina prohlížečů totiž umožňuje zobrazit informace o navázaném spojení.
Odsud stačí potřebné informace jen zkopírovat.
V prohlížečích založených na jádře Chromium lze se lze pomocí klávesy \c|F12| dostat do vývojářského panelu.
V záložce \c|Network| je možné najít příslušný soubor, jehož hlavičku je třeba použít.

\podsekce{Způsoby kontroly změn}
U některých stránek často mění její část, i když sledovaný obsah se nezmění.
Pro omezení kontrolování je proto vhodné užít v zápisu položky \c|diff|.
Ta by měla obsahovat speciální strukturu  popisující způsob ignorování změn\footnote{Případně může obsahovat řetězec \c|\"ignore\"|, který má stejný efekt jako \c|\{\"ignore\":1\}|}.
Tato struktura může obsahovat následující položky:\\
Klíč \c|ignore| s libovolnou hodnotou znamená, že stránka se vůbec nebude kontrolovat na změny.
Toto je vhodné například když se jedná pouze o přihlašovací stránku, která neobsahuje žádaná data.\\
Tag \c|ignoreSector| může obsahovat pole struktur.
Každá z nich může obsahovat informace o jednom nutném vynechání pomocí tagů \c|start|, \c|end| a \c|countOfEnd|,
které znamenají, že text od \c|start| po \c|countOfEnd|-tý výskyt řetězce \c|end| bude při porovnávání vynechán.
Začátek i konec může být definován pomocí regulárního výrazu.
Na přesnou implementaci regulárních výrazů lze nahlédnout do dokumentace Qt na adrese
\url{https://doc.qt.io/qt-5/qregexp.html}.
Tagy \c|end| a \c|countOfEnd| nemusí být uvedeny. V takovém případě je \c|end| nastaveno na konec řádku a \c|countOfEnd| na 1.
% TODO Implementovat regex
\\
Obdobným způsobem lze použít tag \c|onlySector|, který také může obsahovat pole struktur složených z \c|start|, \c|end| a \c|countOfEnd|.
Při použití tohoto tagu budou porovnávány pouze změny v těchto úsecích (budou ignorovány úseky od začátku k prvnímu výskytu, mezi nimi a od posledního na konec).
% TODO Implementace onlySector
\\
Další možností je využít klíče \c|permutation| s libovolnou hodnotou.
Jeho použití znamená, že libovolné permutace znaků budou považovány za stejné.
Toto je vhodné v případě, že se na stránce některé objekty náhodně prohazují.
% TODO Opravit: řádek 71 v conectionthread.cpp
% TODO nahradit permutatin -> permutation 
\\
Tyto omezení porovnávání se provádí v zde uvedeném pořadí.




\podsekce{Skupiny stránek, přihlášení a práce s cookie}
Například v případě, že je nutné pro získáni informací se na nějakou stránku přihlásit, je možné využít v zápisu stránky položku \c|cookie|.
Pomocí ní lze spojit více stránek do skupiny, ve které si stránky mezi sebou ukládají cookies.
Stačí pouze u všech stránek vyplnit stejnou hodnotou tagu \c|cookie|.
Když se některou z stránek nepodaří načíst, v daném průchodu se nebudou načítat ani další stránky ze stejné skupiny.

\sekce{Spuštění kontroly, tabulka změn, informační konzole}
Po spuštění aplikace se zobrazí hlavní okno.
Jelikož aplikace by měla běžet jako služba, okno neobsahuje tlačítko zavřít.
V případě, že uživatel skutečné chce uzavřít aplikaci a tím i zabránit dalším automatickým kontrolám, je možné aplikaci ukončit z menu kliknutím na \c|app| $\rightarrow$ \c|quit|.

Pro manuální spuštění kontroly je možné kliknout na tlačítko \c|Start checking|.
V případě, že již kontrola běží, po kliknutí se ukončí a ihned začne znovu od začátku.
Po kliknutí na \c|Stop checking| se neprodleně prohledávání ukončí.

Při průchodu se nalezené změny ihned zobrazují do tabulky změn.
Tedy i v případě, že je prohledávání ukončeno v průběhu, doposud nalezené změny budou uloženy a zobrazeny.

Aktuální stav průchodu je vidět na stavovém baru  nad tlačítky.
V případě, že bar má červenou barvu, alespoň jedna stránka nebyla při posledním průchodu úspěšně načtena.
Po kliknutí na zaškrtávací políčko \c|view more information| se zobrazí záznam provedených kontrolách a případně důvod jejich neúspěchu.
Všechny chybové hlášky jsou obsaženy na řádcích uvozených několika vykřičníky.
Nejčastěji se může uživatel setkat s těmito chybami:\\
\c|Connection error - TIME OUT| nastane v případě, že načítání stránky bylo moc pomalé a tedy byl překročen časový limit na přístup na jednu stránku.\\ %TODO Opravit conection -> connection aplikaci
\c|Cookie error - cookie not available| se zobrazí v případě, nastala chyba při načítání některé z předcházejících stránek ze stejné cookie skupiny.\\
\c|Connection error - Network access is disabled.| znamená nedostupnost síťového připojení.\\
\c|Connection error - Host not found.| informuje o nedostupnosti daného serveru. S největší pravděpodobností se  jedná o chybě zadanou stránku, nebo byla přesunutá.

V boxu vpravo od tlačítek je možno nastavit automatické spouštění kontrolování.
Do boxu stačí napsat počet minut mezi automatickými kontrolami.
Automatické kontroly lze vypnout napsáním do políčka \c|0|.
Datum a čas poslední úspěšné (tedy takové, ve které se načetly všechny stránky) kontroly je vidět mezi tlačítky boxem automatické aktualizace.
V tomto místě je také vidět čas příští plánované kontroly.

Informace o poslední kontrole a periodě automatické kontrole si aplikace ukládá do souboru databáze v pracovním adresáři aplikace.
Při spuštění si tento soubor načte (pokud existuje).
Nastavení automatické kontroly tedy vydrží i vypnutí a zapnutí aplikace.
V případě, že měla kontrola proběhnout v momentě, kdy byla aplikace vypnutá, proběhne neprodleně po jejím spuštění.

Všechny nalezené změny se zobrazují v tabulce změn umístěné v horní části hlavního okna aplikace.
O každé změně se zobrazí řádek obsahující jméno stránky, čas detekování změny a jméno souboru, v němž je uložena aktuální verze.
V případě, že uživatel chce data setřídit podle některé z těchto položek, může tak učinit kliknutím na hlavičku daného sloupce tabulky.
Kliknutím na jméno souboru se ve výchozím prohlížeči otevře daná verze stránky.
Aby bylo zajištěno správné fungovaní zobrazení, jsou všechny relativní odkazy (v rámci serveru) přepsány na absolutní doplněním názvu.
Před všechny odkazy v atributech \c|href| s \c|src|, které neobsahují absolutní cestu je doplněn název serveru a složky.
Díky tomu se při kontrole načítá pouze samostatná stránka, ale při zobrazení se načtou i obrázky, styly a další odkazované elementy.

V případě, že už si uživatel danou změnu prohlédl, kliknutím na buňku v sloupci \c|Delete| ji může z tabulky změn odstranit.
Tímto odstraněním nedojde k smazání záznamu o změně ani odstranění souboru s danou verzi stránky.
Změna se již nabude zobrazovat v tabulce změn.
Jiným způsobem odstraněni je vybrat jeden nebo několik řádků a pak kliknout na tlačítko \c|Hide| těsné pod tabulkou.

Vybráním řádků a kliknutím na tlačítko \c|Delete| dojde k smazání vybraných záznamů změn včetně souborů obsahujících dané verze stránek.

Všechny záznamy o změnách se ukládají do databáze a při spuštění aplikace se načtou všechny neskryté záznamy. 

\sekce{Historie změn a její procházení}
Zobrazení již odstraněných záznamů z~tabulky změn lze pomocí nástroje historie.
Otevřít ho lze pomocí menu v~hlavním okně kliknutím na \c|pages| $\rightarrow$ \c|history|.
% isdo Ten nadpis co není ani tučně je fakt špatnej
Po kliknutí se otevře další okno obsahující formulář na dotaz a pod ním prázdnou tabulku se stejnou strukturou jako původní tabulka změn.

Dotazy do historie se zadávají formou dotazu do databáze. 
Do políčka lze doplnit podmínky hledání a po kliknutí na talčítko \c|Load| se načte tabulka odpovídajících záznamů. 
Tato načtení je jednorázové, tedy v~případě, že se například vygeneruje další záznam, tabulka historie se sama neobnoví.
Obnovení lze vynutit opětovným stiskem tlačítka \c|Load|.
Jedinou výjimkou jsou změny provedené v~daném okně historie -- smazání a schování.
Pod tabulkou historie jsou stejné ovládací prvky jako pod tabulkou změn.
Tlačítko \c|Delete| smaže vybrané záznamy, ovšem tlačítko \c|Hide| funguje pouze na zatím neodstraněné záznamy z~tabulky změn a to tak, že je schová, ovšem pouze v~tabulce změn.
V~tabulce historie jsou schované záznamy zobrazeny červeně bez vyplněného sloupce \c|Delete|.

Do dotazů je možné psát libovolný SQL dotaz.
Zdrojová tabulka by měla být \c|newPage| obsahující informace o~všech změnách. Její struktura je:
\begin{itemize}
	\item \c|id INTEGER PRIMARY KEY AUTOINCREMENT| -- jednoznačny identifikátor záznamu
	\item \c|pageName VARCHAR| -- identifikátor stránky (podle json zdrojového souboru)
	\item \c|time DATETIME| -- čas zachycení dané změny
	\item \c|fileName VARCHAR| -- jméno souboru obsahujícího danou změnu
	\item \c|del bit| -- obsahuje 1 v~případě, že stránka byla odstraněna z~tabulky změn.
		% TODO del zní fakt blbě
\end{itemize}
Dotaz by ovšem měl vracet tabulku o~stejných sloupcích jako má \c|newPage|, aby mohl být správně graficky vykreslen.
V~opačném případě je chování aplikace nedefinované.
Databáze je implementovaná pomocí SQLite.
Podrobná dokumentace je uveden na adrese \url{https://www.sqlite.org/lang_select.html}.
\obr{history}{Okno přístupu do historie.}

\sekce{Grafické porovnávání verzí stránek}
V~případě, že uživatel chce vidět rozdíly mezi dvěma verzemi stránek (popřípadě i mezi dvěma stránkami) je možné využít nastroje porovnání.
Tento nástroj lze spustit kliknutím na tlačítko \c|Compare difference| pod tabulkou změn nebo pod tabulkou historie.
Před kliknutím je potřeba vybrat jeden nebo dva řádky v~příslušné tabulce.
V~případě, že jsou vybrány dva záznamy, provede se porovnání jejich souborů.
V~případě, že je vybrána pouze jedna stránka, provede se pouze porovnání mezi zvolenou verzí zvolené stránky a předchozí verzí dané stránky.
Zobrazí se tedy rozdíly, které byly detekovány v~daný moment.

Při použití porovnání se ve výchozím prohlížeči otevře stránka obsahující dva rámy obsahující obě porovnávané stránky.
Dále se také otevře ovládací okno.
V~ovládacím okně je možno nastavit dva přepínače:\\
\c|View source code|, Při jeho zaškrtnutí se v~prohlížeči zobrazí místo stránek jejich zdrojové kódy.
Přepínač \c|Two page in one| místo dvou rámů zobrazí jen jeden, do kterého zobrazí sjednocení obou stránek.
Tato funkce je zatím pouze v~testovací verzi a může dojít k~nesprávnému propojení stránek. Stabilní je pouze při propojení s~zobrazením zdrojového kódu.
Pro vytvoření grafického výstupu do prohlížeče dle zadaných kritérii je nutné zmáčknout tlačítko \c|Generate|.
K~znovuotevření stránky v~prohlížeči slouží tlačítko \c|Open|.

V~jakékoliv verzi porovnávání se graficky zobrazují změny.
V~režimu se dvěma rámy se v~každém z~nich zobrazí přebývající text s~červeným zvýrazněním.
V~případě režimu jednoho rámu se přibitý text zobrazí se zeleným podbarvením a odstraněný text červeně.
V~režimu zdrojového kódu se změny zobrazí na čemkoliv -- textu i html elementu či jiném prvku stránky.
Když je ovšem porovnání formou stránek, jsou zvýrazněny pouze změny v~čistém textu.

Porovnávání stránek se provádí na úrovni slov a tagů.
Aplikace se snaží najít jejich zobrazení na sebe s~největším překryvem,  tedy nejmenším počtem rozdílů.
\obrsize{compare}{Porovnávání stránek otevřené v~aplikaci Chromium.}{15}

%\sekce{Klávesové zkratky}

\kapitola{Implementace aplikace}

\sekce{Použitý software}
\podsekce{Qt}
Qt je multiplatformní knihovna pro c++ umožnující nejen tvorbu grafických aplikací, ale i práci se soubory, obrázky či sítovým připojením.
Byla nedílnou součástí vyvíjené aplikace.
Knihovna podporuje běh samotného jádra aplikace -- načítání webových souborů a také se stará o grafické zobrazování dat.
Nedílnou součástí jsou její funkce při většině výpočtů aplikace.
\podsekce{Qt Creator}
Qt Creaator je vývojové prostředí určené primárně pro vývoj aplikací využívajících knihovny Qt.
Programátorovi nabízí zvýrazňování syntaxe a doplňování názvů.
Umožňuje také jednoduché kompilování, spouštění a debugování aplikace pomocí jednoduše ovladatelných nástrojů.
Dále také umožňuje jednoduché ovládání gitu pomocí zabudovaných nástrojů.
V tomto prostředí vznikla většina zdrojových kódů aplikace.
\obr{qtcreator}{Vývojové prostředí Qt Creator}
\podsekce{Git}
Git je systém na správu verzí využitý při vývoji této aplikace.
Umožňuje oddělení jednotlivých úkonů při vývoji, jejich zdokumentování a možnost zobrazení jejich historie.
Dále nabízí pomocí vzdálených repositářů jednoduchý způsob, jak aplikaci nahrát na web a zpřístupnit uživatelům.

\sekce{Objektový model, rozdělení problému}
\sekce{Pozadí aplikace}

\sekce{Grafické uživatelské rozhraní}

\begin{literatura}
\citace{ucebnice}{Novák, 1991}{\autor{Novák, J.} a~kol.
\nazev{Konstrukční vlastnosti ocelí třídy 18}. Praha:
SNTL, 1991. 439~s. ISBN 80-8432-289-9.}
	\citace{qt}{The Qt Company, 2019}{The Qt Company, 2019. Qt 5.12. \nazev{Qt} [online]. The Qt Company [Cit. 31.11.2019]. Dostupné z: \url{https://doc.qt.io/qt-5.12/index.html}}
	\citace{sqlite}{Hipp, 2000}{Hipp, D. R. 2000.  Documentation. \nazev{SQLite} [online]. SQLite [Cit. 31.11.2019]. Dostupné z: \url{https://www.sqlite.org/docs.html}}
	\citace{qt-abc}{Watzke, 2010}{Watzke, D. 2010.  Seriál: Qt 4 - psaní grafických programů. \nazev{AbcLinuxu} [online]. AbcLinuxu [Cit. 31.11.2019]. Dostupné z: \url{http://www.abclinuxu.cz/serialy/qt-4-psani-grafickych-programu}}
	\citace{git}{Chacon, 2009}{\autor{Chacon, S.} \nazev{Pro Git}. Praha: CZ.NIC, 2009, ISBN: 978-80-904248-1-4.}

% Dominantní tvůrce, rok. Název vedlejší webové stránky. Název hlavní webové stránky / webového sídla: podnázev hlavní webové stránky [typ nosiče]. Další tvůrce. Místo: nakladatel, datum publikování. Datum aktualizace [cit. datum citování]. Dostupné z: DOI nebo adresa. Path: cesta.
\end{literatura}

\end{document}
