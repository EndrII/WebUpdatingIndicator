\pdfoutput=1
\documentclass[twoside,12pt]{article}%
\usepackage[utf8]{inputenc} % použito v případě jiného kódování
% aktuální kódování: utf8
\usepackage{dipp}
\usepackage[czech]{babel}
\usepackage{graphicx}


% ~~~~~ PACKAGE ~~~~~
\usepackage[utf8]{inputenc}  % základní kódování
\usepackage[czech]{babel}    % české nastavení
\usepackage{amsmath, amssymb}   % matematické simboly
% amsfonts
\usepackage{graphicx}  % barvy a jiné
\usepackage{lastpage}   % zjištění počtu stránek
\usepackage{calc}  % snazší výpočty délek
\usepackage{refcount} % převod reference na číslo - dvojstraný režim - dělení 2
\usepackage{fancyhdr} % hlavičky
\usepackage{etoolbox}  % změna geimetrie na určité stránky
\usepackage{pgfplots} % geogebra
\usepackage[final]{pdfpages} % geogebra v pdf
%mathrsfs, pdf, tikz, mathrsfs
\usepackage{listingsutf8} % zdrojové kódy
\usepackage{hyperref}
% ~~~~~ MATEMATICKÉ ~~~~~
\renewcommand{\(}{\left(}
\renewcommand{\)}{\right)}
\renewcommand{\[}{\left[}
\renewcommand{\]}{\right]}
\def\MATHINFO{}
  \newcommand{\mathinfo}[2]{\if ^\MATHINFO^ \underbrace{\def\MATHINFO{1} {#1} \def\MATHINFO{} }_{\text{#2}} \else \overbrace{\def\MATHINFO{} {#1} \def\MATHINFO{1}}^{\text{#2}} \fi}
\newcommand{\Mathinfo}[1]{\qquad \qquad \text{({#1})}}

\newcommand{\tg}{\mathrm{tg}}
\newcommand{\cotg}{\mathrm{cotg}}
\newcommand{\sgn}{\mathrm{sgn}}
\renewcommand{\angle}{\sphericalangle}
\newcommand{\degre}{\ensuremath{^\circ}}
\def\d{\degre}
\newcommand{\pri}{\overleftrightarrow}
\newcommand{\ppri}{\overrightarrow}
\def\imp{\Rightarrow}
\def\Imp{\quad\imp\quad}
\def\ekv{\Leftrightarrow}
\def\Ekv{\quad\ekv\quad}
\def\rimp{\Leftarrow}
\def\RImp{\quad\rimp\quad}
\renewcommand{\*}{\cdot{}}
\def\N{\mathbb{N}}
\def\Z{\mathbb{Z}}
\def\Q{\mathbb{Q}}
\def\R{\mathbb{R}}
\def\C{\mathbb{C}}
\def\E{\mathbb{E}}
\def\P{\mathbb{P}}
\def\US{\leftrightarrow{}}

\newcommand{\eqnl}[1][]{{#1}$$\\[-16px]$${}{#1}}

\newcommand{\derivation}[2]{\frac{\partial}{\partial {#1}}\({#2}\)}
\renewcommand{\j}[2][]{\ \mathrm{\if ^#1^ {#2} \else \frac{{#1}}{{#2}} \fi}}

\def\f{\frac}
\def\uv#1{„#1“}

\newcounter{lemmaCounter}[section]
\renewcommand{\thelemmaCounter}{\arabic{lemmaCounter}}
\newtheorem{lemmaBlock}[lemmaCounter]{}
\newenvironment{lemma}[3][]
{

	\refstepcounter{lemmaCounter}
		\label{#2}
	\textbf{Lemma \thelemmaCounter\if^#1^\else \ ({#1})\fi:}
	\emph{{#3}}\\{}
	Důkaz: 
}
{ 
\hfill $\Box$

}

% ~~~~~ INFORMATICKÉ ~~~~~
\newcommand{\refcodeline}[1]{{\footnotesize Viz \ref{#1}. řádek kódu na straně \pageref{#1}.}}
\newcommand{\refcodeblock}[1]{{\footnotesize Viz \ref{#1}.-\ref{#1-end}. řádek kódu na straně \pageref{#1}.}}
\definecolor {lightGrey}{RGB}{250,250,250}
\newcommand{\Onotation}[1]{\ifmmode\mathcal{O}({#1})\else$\mathcal{O}({#1})$\fi}
\renewcommand{\O}{\Onotation}
\newcommand{\codeF}[1]{\lstinputlisting[ numbers = left, numberstyle = \tiny ,frame = shadowbox, backgroundcolor =\color{lightGrey}, showstringspaces=false]{#1} }
\newcommand{\codeMain}{\codeF{main.cpp}}
\lstset{language = C++,basicstyle =  \small \ttfamily, keywordstyle = \emph,commentstyle = \rmfamily ,  backgroundcolor =\color{lightGrey}, showstringspaces=false, escapeinside ={///}{/}, extendedchars=true, 
literate={á}{{\'a}}1 {í}{{\'i}}1 {é}{{\'e}}1 {ý}{{\'y}}1 {ú}{{\'u}}1 {ó}{{\'o}}1 {ě}{{\v{e}}}1 {š}{{\v{s}}}1 {č}{{\v{c}}}1 {ř}{{\v{r}}}1 {ž}{{\v{z}}}1 {ď}{{\v{d}}}1 {ť}{{\v{t}}}1 {ň}{{\v{n}}}1 {ů}{{\r{u}}}1 {Á}{{\'A}}1 {Í}{{\'I}}1 {É}{{\'E}}1 {Ý}{{\'Y}}1 {Ú}{{\'U}}1 {Ó}{{\'O}}1 {Ě}{{\v{E}}}1 {Š}{{\v{S}}}1 {Č}{{\v{C}}}1 {Ř}{{\v{R}}}1 {Ž}{{\v{Z}}}1 {Ď}{{\v{D}}}1 {Ť}{{\v{T}}}1 {Ň}{{\v{N}}}1 {Ů}{{\r{U}}}1  ,}
\def\c{\lstinline}




\makeatother
\begin{document}

\def\,{\penalty10000\hskip.25em}
\pagestyle{headings}

\cislovat{2}
\bakalarska

\titul{Vývoj aplikace na kontrolování změn na webových stránkách}{Jiří Kalvoda}{Mgr. Marek Blaha}{Blansko 2020}

\podekovani{TODO poděkování}

\prohlaseni{Prohlašuji, že jsem tuto práci vyřešil samostatně
s~použitím literatury, kterou uvádím v~seznamu}{V~Blansku dne \today}

\abstract{Kalvoda, J. }{Development an aplication for checking changes on web pages}{abstract}

\abstrakt{Kalvoda, J. \NAZEV}
{Tato závěrečná práce popisuje vývoj a použití aplikace na monitorování změn na webových stránkách.
Aplikace je vyvíjena v jazyce c++ pomocí knihovny Qt. Díky tomu se jedná o multiplatformní software.
Je dostupná včetně zdrojového kódu pod licencí GNU LGPL. Tato závěrečná práce obsahuje popis jejího fungování, implementace a použitého softwaru při jejím vývoji.}

\obsah

\kapitola{Úvod a~cíl práce}
\sekce{Úvod do problematiky}

Webové stránky se mohou neustále měnit, proto je dobré automaticky monitorovat jejich aktualizace.
Tato aplikace umožňuje automatizovat tento problém a tím uživateli ušetřit čas a eliminovat lidský chybový faktor.

Aplikace podporuje různé tolerance při načítání a porovnávání změn stránek.
Například na stránce, kde se část neustále mění, mohu tuto část vypustit, nebo přímo porovnávat jen nějaké části a podobně. Díky práci s cookies je možné navázat i složitější spojení se serverem a provést definovanou sekvenci úkolů (např. přihlásit se a načíst nějaký soukromý obsah). Historie stránek se může ukládat a pak lze v~ní vyhledávat a zjišťovat rozdíly mezi verzemi pomocí grafického porovnávání napojeného na uživatelův oblíbený prohlížeč.

\sekce{Cíl práce}

Cílem této práce je vyvinout funkční aplikaci umožnující zjišťování aktualizací, archivaci a porovnávání webových stránek (případně i jiných dokumentů) a publikovat ji cílovým uživatelům na různých operačních systémech.
K aplikaci také bude vypracována rozsáhlá uživatelská i technická dokumentace, která umožní její další vývoj.
Umožním tedy dalším programátorům tuto aplikaci pohodlně modifikovat a upravovat dle svých potřeb.
Cílem této práce je také aplikaci rozšířit mezi skupinu testovacích uživatelů a použít jejich připomínky a problémy k dalšímu vývoji a stabilizaci aplikace.

\kapitola{Přehled literatury}

\kapitola{Popis fungování a ovládání aplikace}
\sekce{Instalace}
\podsekce{Linux}
\podsekce{Windows}
\podsekce{macOS}


\sekce{Seznam stránek na kontrolu}
Při spuštění si aplikace načte seznam stránek ke kontrole.
Ten je obsažen v souboru \c|pages.json|, který musí být umístěn v adresáři aplikace (respektive v adresáři, kde se aplikace spouští).
Soubor musí být validní json.
V případě, že soubor neexistuje nebo není validní, aplikace vypíše upozornění. % TODO naimplementovat to
Očekává se, že soubor obsahuje pole struktur.
Každá z nich obsahuje informace o jedné stránce, která se má kontrolovat.
\podsekce{Adresa stránky a název}
Každou stránku je nutné pojmenovat jednoznačným identifikátorem.
Proto je nutné u každé stránky definovat položku označenou klíčem \c|name|.
Toto jméno se pak zobrazuje v seznamu změn a také je nutné při vyhledávání v historii.
Jméno se také používá jako identifikátor v databázi i jako identifikátor souborů jednotlivých verzí stránky.

Dále je nutné u každé stránky definovat její umístění na webu, tedy její adresu.
K tomu slouží položky \c|server|, \c|dir|, \c|file|.
Ovšem není zapotřebí definovat všechny tyto položky.
Pod klíčem \c|server| by měla být definována adresa serveru, na kterém je požadovaná stránka včetně přístupového protokolu.
Tedy například \c|https://is.jaroska.cz| nebo pomocí ip adresy může zápis vypadat následovně \c|http://195.178.65.1|
V případě, že není požadovaná stránka na serveru umísténa v jeho kořenové složce, je doporučeno název složky (popřípadě celou cestu několika zanořených složek) umístit do položky \c|dir|.
\c|file| obsahuje jméno požadovaného souboru včetně přípony
V případě, že je požadován přístup základní soubor (index) na dané adrese, není potřeba \c|file| uvádět.
Dále zde také můžou být obsaženy parametry stránky, které se předávají metodou \c|GET|.
Stačí je uvézt za otazník.
Takový zápis může vypadat následovně: \c|index.php?akce=42&akcicka=0|.

V případě, že je nutné předat stránce informace pomocí protokolu \c|POST| (typicky při přihlašování na stránky), 
je možné v zápisu stránky užít klíče \c|post|.
%\c|formUsername=HugoKokoska&formPassword=BezpecneHeslo|.

Nejjednodušší způsob, jak zjistit tyto informace pro požadovanou stránku je využít prohlížeč.
Většina prohlížečů totiž umožňuje zobrazit informace o navázaném spojení.
Odsud stačí potřebné informace jen zkopírovat.
V prohlížečích založených na jádře Chromium lze se lze pomocí klávesy \c|F12| dostat do vývojářského panelu.
V záložce \c|Network| je možné najít příslušný soubor, jehož hlavičku je třeba použít.

\podsekce{Způsoby kontroly změn}
U některých stránek často mění její část, i když sledovaný obsah se nezmění.
Pro omezení kontrolování je proto vhodné užít v zápisu položky \c|diff|.
Ta by měla obsahovat speciální strukturu  popisující způsob ignorování změn\footnote{Případně může obsahovat řetězec \c|\"ignore\"|, který má stejný efekt jako \c|\{\"ignore\":1\}|}.
Tato struktura může obsahovat následující položky:\\
Klíč \c|ignore| s libovolnou hodnotou znamená, že stránka se vůbec nebude kontrolovat na změny.
Toto je vhodné například když se jedná pouze o přihlašovací stránku, která neobsahuje žádaná data.\\
Tag \c|ignoreSector| může obsahovat pole struktur.
Každá z nich může obsahovat informace o jednom nutném vynechání pomocí tagů \c|start|, \c|end| a \c|countOfEnd|,
které znamenají, že text od \c|start| po \c|countOfEnd|-tý výskyt řetězce \c|end| bude při porovnávání vynechán.
Začátek i konec může být definován pomocí regulárního výrazu.
Na přesnou implementaci regulárních výrazů lze nahlédnout do dokumentace Qt na adrese
\url{https://doc.qt.io/qt-5/qregexp.html}.
Tagy \c|end| a \c|countOfEnd| nemusí být uvedeny. V takovém případě je \c|end| nastaveno na konec řádku a \c|countOfEnd| na 1.
% TODO Implementovat regex
\\
Obdobným způsobem lze použít tag \c|onlySector|, který také může obsahovat pole struktur složených z \c|start|, \c|end| a \c|countOfEnd|.
Při použití tohoto tagu budou porovnávány pouze změny v těchto úsecích (budou ignorovány úseky od začátku k prvnímu výskytu, mezi nimi a od posledního na konec).
% TODO Implementace onlySector
\\
Další možností je využít klíče \c|permutation| s libovolnou hodnotou.
Jeho použití znamená, že libovolné permutace znaků budou považovány za stejné.
Toto je vhodné v případě, že se na stránce některé objekty náhodně prohazují.
% TODO Opravit: řádek 71 v conectionthread.cpp
% TODO nahradit permutatin -> permutation 
\\
Tyto omezení porovnávání se provádí v zde uvedeném pořadí.




\podsekce{Skupiny stránek, přihlášení a práce s cookie}
Například v případě, že je nutné pro získáni informací se na nějakou stránku přihlásit, je možné využít v zápisu stránky položku \c|cookie|.
Pomocí ní lze spojit více stránek do skupiny, ve které si stránky mezi sebou ukládají cookies.
Stačí pouze u všech stránek vyplnit stejnou hodnotou tagu \c|cookie|.
Když se některou z stránek nepodaří načíst, v daném průchodu se nebudou načítat ani další stránky ze stejné skupiny.


\sekce{Spuštění kontroly, tabulka změn, informační konzole}

\sekce{Otevření lokální kopie stránky, její napojení}

\sekce{Historie změn a její procházení}

\sekce{Grafické porovnávání verzí stránek}

\sekce{Klávesové zkratky}

\kapitola{Implementace aplikace}

\sekce{Použitý software}
\podsekce{Qt}
\podsekce{Git}

\sekce{Objektový model, rozdělení problému}
\sekce{Pozadí aplikace}

\sekce{Grafické uživatelské rozhraní}


\end{document}
