
\sekce{Seznam stránek na kontrolu}
Při spuštění si aplikace načte seznam stránek ke kontrole.
Ten je obsažen v~souboru \c|pages.json|, který musí být umístěn v~adresáři aplikace (respektive v~adresáři, kde se aplikace spouští).
Soubor musí být validní json.
V~případě, že soubor neexistuje nebo není validní, aplikace vypíše upozornění. % isdo naimplementovat to
Očekává se, že soubor obsahuje pole struktur.
Každá z~nich obsahuje informace o~jedné stránce, která se má kontrolovat.
\obr{wa-json}{Chybný vstupní soubor.}
\podsekce{Adresa stránky a název}
Každou stránku je nutné pojmenovat jednoznačným identifikátorem.
Proto je nutné u~každé stránky definovat položku označenou klíčem \c|name|.
Toto jméno se pak zobrazuje v~seznamu změn a také je nutné při vyhledávání v~historii.
Také se používá jako identifikátor v~databázi i jako identifikátor souborů jednotlivých verzí stránky.

Dále je nutné u~každé stránky definovat její umístění na webu, tedy její adresu.
K~tomu slouží položky \c|server|, \c|dir|, \c|file|.
Ovšem není zapotřebí definovat všechny tyto položky.
Pod klíčem \c|server| by měla být definována adresa serveru, na kterém je požadovaná stránka včetně přístupového protokolu.
Tedy například \c|https://is.jaroska.cz| nebo pomocí ip adresy může zápis vypadat následovně: \c|http://195.178.65.1|.
V~případě, že není požadovaná stránka na serveru umístěna v~jeho kořenové složce, je doporučeno název složky (popřípadě celou cestu několika zanořených složek) umístit do položky \c|dir|.
\c|file| obsahuje jméno požadovaného souboru včetně přípony.
V~případě, že je požadován přístup na základní soubor (index) na dané adrese, není potřeba \c|file| uvádět.
Dále zde také mohou být obsaženy parametry stránky, které se předávají metodou \c|GET|.
Stačí je uvézt za otazník.
Takový zápis může vypadat následovně: \c|index.php?akce=42&akcicka=0|.

V~případě, že je nutné předat stránce informace pomocí protokolu \c|POST| (typicky při přihlašování na stránky), 
je možné v~zápisu stránky užít klíče \c|post|.
%\c|formUsername=HugoKokoska&formPassword=BezpecneHeslo|.

Nejjednodušší způsob, jak zjistit tyto informace pro požadovanou stránku, je využít prohlížeč.
Většina prohlížečů umožňuje zobrazit informace o~navázaném spojení.
Odsud stačí potřebné informace jen zkopírovat.
V~prohlížečích založených na jádře Chromium se lze pomocí klávesy \c|F12| dostat do vývojářského panelu.
V~záložce \c|Network| je možné najít příslušný soubor, jehož hlavičku je třeba použít.

\podsekce{Způsoby kontroly změn}
U~některých stránek se často mění jejich část, i když se sledovaný obsah nezmění.
Pro omezení kontrolování je proto vhodné užít v~zápisu položky \c|diff|.
Ta by měla obsahovat speciální strukturu  popisující způsob ignorování změn\footnote{Případně může obsahovat řetězec \c|\"ignore\"|, který má stejný efekt jako \c|\{\"ignore\":1\}|}.
Tato struktura může obsahovat následující položky:\\
Klíč \c|ignore| s~libovolnou hodnotou znamená, že stránka se vůbec nebude kontrolovat na změny.
Toto je vhodné například když se jedná pouze o~přihlašovací stránku, která neobsahuje žádaná data.\\
Tag \c|ignoreSector| může obsahovat pole struktur.
Každá z~nich může obsahovat informace o~jednom nutném vynechání pomocí tagů \c|start|, \c|end| a \c|countOfEnd|,
které znamenají, že text od \c|start| po \c|countOfEnd|-tý výskyt řetězce \c|end| bude při porovnávání vynechán.
Tagy \c|end| a \c|countOfEnd| nemusí být uvedeny. V~takovém případě je \c|end| nastaveno na konec řádku a \c|countOfEnd| na 1.
\\
Obdobným způsobem lze použít tag \c|onlySector|, který také může obsahovat pole struktur složených ze \c|start|, \c|end| a \c|countOfEnd|.
Při použití tohoto tagu budou porovnávány pouze změny v~těchto úsecích (budou ignorovány úseky od začátku k~prvnímu výskytu, mezi nimi a od posledního na konec).
\\
Další možností je využít klíče \c|permutation| s~libovolnou hodnotou.
Jeho použití znamená, že libovolné permutace znaků budou považovány za stejné.
Toto je vhodné v~případě, že se na stránce některé objekty náhodně prohazují.
% isdo Opravit: řádek 71 v conectionthread.cpp
% isdo nahradit permutatin -> permutation 
\\
Tato omezení porovnávání se provádí ve zde uvedeném pořadí.




\podsekce{Skupiny stránek, přihlášení a práce s~cookie}
Například v~případě, že je nutné se pro získáni informací na nějakou stránku přihlásit, je možné využít v~zápisu stránky položku \c|cookie|.
Pomocí ní lze spojit více stránek do skupiny, ve které si stránky mezi sebou ukládají cookies.
Stačí pouze u~všech stránek vyplnit stejnou hodnotou tagu \c|cookie|.
Když se některou ze stránek nepodaří načíst, v~daném průchodu se nebudou načítat ani další stránky ze stejné skupiny.
\obrazek{\codeF{priklad.json}}{Příklad validního souboru stránek na kontrolu.}{json}
