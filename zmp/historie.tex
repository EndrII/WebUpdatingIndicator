\sekce{Historie změn a její procházení}
Zobrazení již odstraněných záznamů z tabulky změn lze pomocí nástroje historie.
Otevřít ho lze pomocí menu v hlavním okně kliknutím na \c|pages| $\rightarrow$ \c|history|.
% isdo Ten nadpis co není ani tučně je fakt špatnej
Po kliknutí se otevře další okno obsahující formulář na dotaz a pod ním prázdnou tabulku se stejnou strukturou jako původní tabulka změn.

Dotazy do historie se zadávají formou dotazu do databáze. 
Do políčka lze doplnit podmínky hledání a po kliknutí na talčítko \c|Load| se načte tabulka odpovídajících záznamů. 
Tato načtení je jednorázové, tedy v případě, že se například vygeneruje další záznam, tabulka historie se sama neobnoví.
Obnoveni lze vynutit opětovným stiskem tlačítka \c|Load|.
Jedinou výjimkou jsou změny provedené v daném okně historie -- smazání a schování.
Pod tabulkou historie jsou stejné ovládací prvky jako pod tabulkou změn.
Tlačítko \c|Delete| smaže vybrané záznamy, ovsem tlačítko \c|Hide| funguje pouze na zatím neodstraněné záznamy z tabulky změn a to tak, že je schová  v tabulce změn.
V tabulce historie jsou schované záznamy zobrazeny červeně bez vyplněného sloupce \c|Delete|.

Do dotazů je možné psát libovolný SQL dotaz.
Zdrojová tabulka by měla být \c|newPage| obsahující informace o všech změnách. Její struktura je:
\begin{itemize}
	\item \c|id INTEGER PRIMARY KEY AUTOINCREMENT| -- jednoznačny identifikátor záznamu
	\item \c|pageName VARCHAR| -- identifikátor stránky (podle json zdrojového souboru)
	\item \c|time DATETIME| -- čas zachycení dané změny
	\item \c|fileName VARCHAR| -- jméno souboru obsahujícího danou změnu
	\item \c|del bit| -- obsahuje 1 v případě, že stránka byla odstraněna z tabulky změn.
		% TODO del zní fakt blbě
\end{itemize}
Dotaz by ovšem měl vracet tabulku o stejných sloupcích jako má \c|newPage|, aby mohl být správně graficky vykreslen.
V opačném případě je chování aplikace nedefinované.
Databáze je implementovaná pomocí \c|SQLITE|.
Podrobná dokumentace je uveden na adrese \url{https://www.sqlite.org/lang_select.html}.
