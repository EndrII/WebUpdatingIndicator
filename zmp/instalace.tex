\sekce{Instalace}
\podsekce{Kompilace ze zdrojového kódu}
Způsobem, jak aplikaci nainstalovat na většině používaných operačních systémů, je kompilace ze zdrojového kódu.
Zdrojové kódy v~aktuální stabilní verze je možné stáhnout z~\url{gitlab.com/JiriKalvoda/webupdatingindicator/tree/master}.
V~případě, že má uživatel zájem o~aktuálně nejnovější funkce, je možné použít testovací verzi produktu dostupnou z~\url{gitlab.com/JiriKalvoda/webupdatingindicator/tree/Test}.
Soubory lze stáhnout pomocí webového rozhraní a nebo je lze naklonovat s~použitím gitu.
Aplikaci pak lze zkompilovat za použití Qt knihoven.
Nejsnazší způsob je využít aplikace Qt Creator. Pomocí ní stačí otevřít soubor \c|WebUpdatingIndicator.pro| a v~levém rohu aplikace kliknout na tlačítko pro kompilaci.
Tímto způsobem by měla vzniknout samostatně spustitelná aplikace, kterou stačí umístit do požadované složky a v~ní ji spouštět.
Podporovaná by měla být libovolná verze Qt větší než 5.4.
Pro vývoj se používá Qt 5.12.5.

V~případě, že uživatel nechce provádět kompilaci ze zdrojového kódu, pro základní architektury a operační systémy je možné využít již zkompilované varianty.
Ty jsou dostupné na adrese \url{gitlab.com/JiriKalvoda/webupdatingindicator-install/tree/master}
\podsekce{Linux}
Na operačních systémech postavených na jádře Linuxu stačí pouze stáhnout a rozzipovat složku s~programem do uživatelem zvoleného adresáře.
Pracovní adresář aplikace je pak ten, ze kterého se aplikace spouští (nemusí tedy být shodný s~adresářem ve kterém je umístěna aplikace).
Pro jednodušší spouštění je dobré vytvořit bash script, který bude obsahovat přepnutí polohy do pracovního adresáře a spuštění aplikace.
Příklad takového skriptu je umístěn v~stahovací složce pod názvem \c|run.sh|.
Pro možnost spouštění aplikace z~menu či pomocí přímého příkazu je možné tento skript umístit do adresáře \c|usr/bin|.

V~případě užívání správce oken i3 je vhodné nastavit, aby se okna porovnávání stránek zobrazovali jako plovoucí.
Toho lze docílit přidáním řádku \c|for_window [title="WebUpdatingIndicator compare"] floating enable| do konfiguračního souboru i3 umístěného v~\c|~/.config/i3/config|.
Pro snazší spouštění aplikace je také vhodné nadefinovat klávesovou zkratku.
Případně je možně vyhradit aplikaci speciální pracovní plochu a definovat její spuštění a přepnutí na danou plochu pomocí příkazů (nastaví spuštění na \c|$mod+Shift,| a zobrazení na \c|$mod+,|):\\
\begin{tabular}{l}
\c|bindsym $mod+Shift+comma workspace WUI;exec WebUpdatingIndicator.sh|\\
\c|bindsym $mod+comma workspace WUI|\\
\end{tabular}
\obr{linux-i3-2}{Aplikace otevřená v~operačním systému Linux Mint 19 se správcem oken i3.}
\obr{linux-cinnamon}{Aplikace otevřená v~operačním systému Linux Mint 19 se správcem oken Cinnamon.}

\podsekce{Windows}
Pro Windows existuje alternativní instalační složka. Jejím stažením a rozzipováním do uživatelem zvolená složky vznikne spustitelná aplikace. 
Jelikož na tomto operačním systému nejsou běžně dostupné potřebné dynamicky linkované knihovny Qt,
instalační složka je přímo obsahuje.
Důsledkem této skutečnosti je, že aplikace zabírá mnohem více diskového prostoru.
Dále je dobré vytvořit zástupce, přes kterého se bude daný program spouštět.
Pří jeho vytvářeni se dá zvolit i pracovní adresář.
Jako ikonku je možné nastavit \c|Logo.ico|.
Zástupce je pak možné umístit například na plochu nebo do start menu.
\obr{windows}{Aplikace otevřená v~operačním systému Windows 10.}

\podsekce{macOS}
Na tomto operačním systému je momentálně možná instalace pouze pomocí kompilace ze zdrojového kódu.


% TODO při inicializaci udělat ty složky.
