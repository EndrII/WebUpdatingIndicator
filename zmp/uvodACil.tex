\kapitola{Úvod a~cíl práce}
\sekce{Úvod do problematiky}

Webové stránky se mohou neustále měnit, proto je dobré automaticky monitorovat jejich aktualizace.
Tato aplikace umožňuje automatizovat tento problém a tím uživateli ušetřit čas a eliminovat lidský chybový faktor.

Aplikace podporuje různé tolerance při načítání a porovnávání změn stránek.
Například na stránce, kde se část neustále mění, mohu tuto část vypustit, nebo přímo porovnávat jen nějaké části nebo lze tolerovat prohazování prvků stránky. Díky práci s~cookies je možné navázat i složitější spojení se serverem a provést definovanou sekvenci úkolů (např. přihlásit se a načíst nějaký soukromý obsah). Historie stránek se může ukládat a pak lze v~ní vyhledávat a zjišťovat rozdíly mezi verzemi pomocí grafického porovnávání napojeného na uživatelův oblíbený prohlížeč.

\sekce{Cíl práce}

Cílem této práce je vyvinout funkční aplikaci umožnující zjišťování aktualizací, archivaci a porovnávání webových stránek (případně i jiných dokumentů) a publikovat ji cílovým uživatelům na různých operačních systémech.
K~aplikaci také bude vypracována rozsáhlá uživatelská i technická dokumentace, která umožní její další vývoj.
Umožním tedy dalším programátorům tuto aplikaci pohodlně modifikovat a upravovat dle svých potřeb.
Cílem této práce je také aplikaci rozšířit mezi skupinu testovacích uživatelů a použít jejich připomínky a problémy k~dalšímu vývoji a stabilizaci aplikace.
\obr{basic_app_small}{Základní okno aplikace.}
